\documentclass[prd,aps,letter,twocolumn,floatfix,notitlepage,nofootinbib]{revtex4-1}

%% packages
\usepackage{graphicx,psfrag}
\usepackage{mathrsfs,amsmath,amsfonts,amssymb, bm}
\usepackage{multirow}
\usepackage{comment,hyperref}
\usepackage{float}
\usepackage{algorithm}
\usepackage{algpseudocode} % in place of algorithmic. Conflicts with revtex4 unless [H] option passed
\usepackage{setspace} % for pleasent spacing in algorithm
\usepackage{times}
\usepackage{ulem}

%% macros
\newcommand{\be}{\begin{equation}}
\newcommand{\ee}{\end{equation}}
\newcommand{\bsube}{\begin{subequations}}
\newcommand{\esube}{\end{subequations}}

%\def\l{\ell}
%\def\blambda{{\bm \lambda}}
%\def\bLambda{{\bm \Lambda}}
%\def\blam{\bar{\lambda}}
%\def\Lam{\Lambda}
%\def\Lamt{\tilde{\Lambda}}
%\def\dLamt{\tilde{\delta{\Lambda}}}
%\def\bmu{{\bm \mu}}

\def\bx{\mathbf{x}}
\def\by{\mathbf{y}}
\def\btheta{\boldsymbol{\theta}}



%% macros for comments
\usepackage{color}
%% colors
\definecolor{cyan}{rgb}{0,0.9,0.9}
\definecolor{orange}{rgb}{0.9,0.5,0}
\definecolor{magenta}{rgb}{1,0,1}
\definecolor{purple}{rgb}{0.8,0.4,0.8}
\definecolor{gray}{rgb}{0.8242,0.8242,0.8242}
%%
\newcommand{\MP}[1]{{\textcolor{blue}{\texttt{MP: #1}} }}
\newcommand{\bs}[1]{{\textcolor{green}{\texttt{SB: #1}} }}
\newcommand{\bl}[1]{{\textcolor{blue}{\texttt{BL: #1}} }}
\newcommand{\cg}[1]{{\textcolor{orange}{{CG: #1}} }}
\newcommand{\cvdb}[1]{{\textcolor{purple}{\texttt{CVDB: #1}} }}
\newcommand{\red}[1]{\textcolor{red}{#1}}
\newcommand{\blue}[1]{\textcolor{blue}{#1}}
%% +++++++++++++++++++++++++++++++++++++++++++++++++++++++
\begin{document}

\title{Surrogate model of aligned-spin binary neutron star waveforms using Gaussian process regression}

\author{
Benjamin D. Lackey, 
Michael P\"{u}rrer, 
Andrea Taracchini,
Sylvain Marsat
}
\affiliation{
Max Planck Institute for Gravitational Physics, Albert Einstein Institute, D-14476 Golm, Germany
} 

\date{\today}

\begin{abstract}

Fast and accurate waveform models are necessary for measuring the properties of inspiraling binary neutron star systems such as the recently discovered event GW170817. The effective one body (EOB) formalism has been used to produce BNS waveforms that include aligned-spin, quadrupolar and octopolar adiabatic tides, as well as the effect of dynamical tides, and the formalism is accurate enough to avoid significant biases in the measured parameters. Unfortunately, current implementations take several minutes to a day to evaluate, and this is too slow for parameter estimation codes that require $10^7$--$10^8$ sequential waveform evaluations. We construct a frequency-domain surrogate of this model that requires less than 0.3~s to evaluate for a waveform with a starting frequency of 10~Hz (0.04~s for a starting frequency of 30~Hz). We first reduce the 12-dimensional intrinsic parameter space of this model to the 5-dimensional parameter space that includes mass ratio, aligned-spins, and quadrupolar tides using universal relations that approximate all matter effects in terms of the leading quadrupolar tidal parameter. We then use Gaussian process regression and active learning to optimally sample the parameter space and construct a surrogate with $\sim 1000$ waveforms instead of the $\gtrsim 10^5$ waveforms that would be required with standard, grid-based interpolation methods. \red{[[Mismatch is ...]] [[It can be used in arbitrary frequency range by matching to analytic TaylorF2.]] [[Improvement to previously published EOB model [cite] by including spin-induced quadrupole moment and prescription for waveform tapering at merger.]] }


\end{abstract}

\pacs{
  % 04.25.D-,   % numerical relativity
  % 
  04.30.Db,   % gravitational wave generation and sources
  04.40.Dg,   % Relativistic stars: structure, stability, and oscillations
  % 04.70.Bw,   % classical black holes
  95.30.Sf,     % relativity and gravitation
  % 
  % 95.30.Lz,   % Hydrodynamics
  % 97.60.Jd    % Neutron stars
  % 97.60.Lf    % black holes (astrophysics)
  % 98.62.Mw    % Infall, accretion, and accretion disks
}

%%
\maketitle


%% ________________________________________________________
\red{Choose point-mass or point-particle.}
\section{Introduction}

The detection of the merging binary neutron star (BNS) system GW170817~\cite{GW170817} by the Advanced LIGO~\cite{Harry2010} and Advanced Virgo~\cite{Acernese2009} gravitational-wave detectors, has demonstrated that the parameters of the binary can be measured with enough precision to answer fundamental physics questions such as identifying BNS mergers as a source of gamma ray bursts [[cite]], measuring the Hubble constant~\cite{GW170817Hubble}, and understanding the structure of neutron stars (NSs) through tidal interactions~\cite{GW170817}. Measuring these parameters accurately will also be necessary to properly interpret the gamma ray burst [[cite]] and optical afterglow [[cite]] that was found in conjunction. Furthermore, an estimate of the BNS merger rate of 320--4740 events Gpc$^{-3}$~yr$^{-1}$ (90\% confidence interval)~\cite{GW170817} indicates that, when second generation detectors reach design sensitivity, including KAGRA~\cite{Somiya2012} and possibly LIGO-India~\cite{IyerSouradeepUnnikrishnan2011}, we could eventually observe a population of tens or hundreds of events after a couple of years of observation~\cite{LIGORate2010}. This will allow us to estimate the NS mass distribution [[cite]], spin distribution [[cite]], and universal equation of state (EOS) that is common to all NSs\cite{DelPozzoLiAgathos2013, LackeyWade2015}. 

These results are obtained using Bayesian analyses that compare the data to parameterized waveform models. To avoid systematic errors highly accurate models are needed, and they must include all relevant physical effects. This includes descriptions of the point-particle interactions, spins, and tides. Prior to the discovery of GW170817, most parameter estimation analyses were done using post-Newtonian (PN) waveform models. These models are currently known up to 3.5 PN order in the point-particle and spin terms, and the leading and next-to-leading order terms for tidal interactions that enter at 5 and 6PN order. Unfortunately, the lack of point-particle terms above 3.5PN order leads to systematic errors in the tidal parameters that are as large as the statistical errors~\cite{Favata2014, YagiYunes2014, WadeCreightonOchsner2014}. Alternative models exist with more accurate representations of the point particle terms. \MP{What are these alternative models besides EOB?} However, they do not always simultaneously include the spin and tidal effects as well. Not including spin effects, for example, can bias the measured tidal parameter for even moderate dimensionless spins of $|\chi_i| \le 0.04$~\cite{Favata2014}. In addition, ignoring tidal interactions can bias the measured mass ratio from equal mass to unequal mass [[cite who?]].

One such alternative to the PN waveforms is the effective one body (EOB) formalism discussed in section~\ref{sec:eob}. It includes several dynamical effects beyond the standard PN adiabatic inspiral evolution and can also be tuned to numerical relativity (NR) simulations. Until recently, however, no version of the EOB formalism included the effects of spin and tides in the same model. Recently two versions have become available that include both tidal effects and aligned spin. These models are being implemented within the LIGO Algorithm Library (\texttt{LAL})~\cite{lal} under the names \texttt{SEOBNRv4T} and \texttt{TEOBResumS}. The \texttt{SEOBNRv4T} model~\cite{Hinderer:2016eia,Steinhoff:2016rfi} used in this paper, includes the tidally induced $\ell=2$ and 3  multipole moments as well as the induced $\ell=2$ and 3 $f$-mode resonances, but does not yet include the spin induced quadrupole moment that can be important for large spins~\cite{Poisson1998, HarryHinderer2017}. This model also agrees with NR simulations of BNS systems to a level consistent with the numerical error of the simulations (less than 1~rad during the last $\sim 10$ orbits before merger)~\cite{DietrichHinderer2017, KiuchiKawaguchiKyutoku2017}. The alternative model, \texttt{TEOBResumS}~\cite{Nagar2017}, uses a slightly different point-particle and spin prescription. For tidal interactions, the model includes the $\ell = 2, 3$, and 4 terms and uses a method for re-summing these expressions. It does not include the effect of dynamical tides from the induced $f$-mode resonances, but does include the spin induced quadrupole moment.

%The alternative version, \texttt{TEOBResumS}~\cite{Nagar2017}, uses a slightly different point-particle and spin prescription. And, in addition, includes the $\ell = 2, 3$, and 4 tidal effects and the spin induced quadrupole moment, but does not include the effect of dynamical tides from the induced $f$-mode resonances. It does, however, use a method for re-summing the expressions for tidal interactions.

In addition to accuracy, useful waveform models must be fast enough to be used in Bayesian parameter estimation algorithms such as Markov chain Monte Carlo (MCMC) and Nested Sampling that currently require $10^7$--$10^8$ sequential waveform evaluations. Waveform evaluation times must therefore be significantly less than 1~s in order to run in less than a month for each event. Most time-domain waveform implementations such as the PN approximants and the EOB models involve solving a set of ordinary differential equations and then Fourier transforming the results into the frequency domain. For PN waveform models, this can take several seconds for a starting frequency of 10~Hz, while for the \texttt{SEOBNRv4T} model discussed here, the equivalent evaluation time can be a day. Significant work has been done to optimize existing implementations of EOB for binary black hole (BBH) systems by a factor of several hundred~\cite{DevineEtienneMcWilliams2016}. However, it is not clear that sufficient speed ups can be achieved for the BNS model discussed here with dynamical tides. 

For the GW170817 event, therefore, a set of frequency-domain analytic or semi-analytic waveform models were used as templates. The first was the stationary phase approximation PN waveform model known as \texttt{TaylorF2}~\cite{BuonannoIyerOchsner2009}. Because of the known systematic errors that result from PN waveforms, two additional models were produced for comparison: \texttt{IMRPhenomD\_NRTides} and \texttt{SEOBNRv4\_ROM\_NRTides}. Both of these models used an identical prescription for tides presented in Ref.~\cite{DietrichBernuzziTichy2017}. At low frequencies, the tidal effect is that of \texttt{TaylorF2}. It then transitions to a fit to the tidal effect measured by a small set of numerical simulations with varying mass ratio and aligned-spins. The point-particle and spin prescriptions were different. For \texttt{IMRPhenomD\_NRTides}, the waveform is based on the phenomenological \texttt{IMRPhenomD} BBH model~\cite{HusaKhanHannam2016, KhanHusaHannam2016}. It uses \texttt{TaylorF2} at low frequencies then transitions to fits to the EOB waveform at mid frequencies and fits to NR simulations at high frequencies. The \texttt{SEOBNRv4\_ROM\_NRTides} model is based on the fast surrogate model \texttt{SEOBNRv4\_ROM}~\cite{Puerrer2015} for the EOB waveform \texttt{SEOBNRv4} for BBH systems~\cite{Bohe:2016gbl}. Both of these models are improvements to \texttt{TaylorF2}, but do not incorporate all known physical effects in a consistent way. We therefore build a fast surrogate that reproduces as accurately as possible \texttt{SEOBNRv4T} in the frequency domain.

Surrogate modeling techniques have had significant success in gravitational-wave data analysis for accelerating the evaluation of waveform models. The essential idea is to construct an interpolating function that approximates a waveform model as a function of frequency (or time) and waveform parameters $\bx$. Previous works have focused on efficiently representing the parameter space of waveforms using a reduced basis of orthonormal functions whose linear combination can reproduce any waveform in the parameter space to some desired accuracy. The two most common approaches have been singular value decomposition~\cite{Puerrer2014, Puerrer2015} and a greedy method that selects a subset of training-set waveforms then orthonormalizes them with the Graham-Schmidt process~\cite{FieldGalleyHesthaven2014, LackeyBernuzziGalley2017, BlackmanFieldGalley2015, BlackmanFieldScheel2017a, BlackmanFieldScheel2017b}. In this work, instead of using a reduced basis to efficiently represent the EOB waveform as a function of frequency, we will instead use the fact that the frequency-domain waveform is approximately known analytically via the TaylorF2 waveform. We then build a surrogate of the difference (or residual) between the EOB waveform and the TaylorF2 waveform. Because this residual is small, extremely high accuracy is not required, and we simply choose to represent the residual as a function of frequency using cubic splines between a small set of frequency nodes.

The final step is to interpolate between waveform parameters $\bx$. Previous work on surrogate models focused on parameter spaces with three or fewer dimensions. This allows one to approximate the waveform as a function of $\bx$ using standard interpolation techniques such as tensor spline~\cite{Puerrer2014, Puerrer2015} or Chebyshev interpolation~\cite{LackeyBernuzziGalley2017}. These interpolation techniques typically require waveforms to be evaluated on a rectangular grid, and thus suffer from the curse of dimensionality; for $N$ points per dimension $d$,  $N^d$ waveforms are needed. For the problem here, we will only be able to reduce the parameter space to five dimensions, so grid based interpolation is unfeasible \MP{I think that this is too strong; Blackman17a also perform 5D TP interpolation which contradicts this.}. Recent work on optimally choosing waveforms for both EOB\MP{We used IMRPhenomD}~\cite{DoctorFarrHolz2017} and NR~\cite{BlackmanFieldScheel2017a, BlackmanFieldScheel2017b} waveforms has shown that there are sufficiently accurate alternatives to grid-based interpolation that do not suffer from the curse of dimensionality.

In this paper we focus on a technique known as Gaussian process regression (GPR)~\cite{RasmussenWilliams2006} that does not require a regular grid. Importantly, GPR also provides a convenient estimate of its own uncertainty. This allows us to iteratively add new points to the training set in such a way as to minimize the interpolation error over the parameter space for a given number of waveforms in the training set. Similar approaches have recently been used to optimally sample waveforms for a 2-dimensional aligned spin BBH surrogate using GPR~\cite{DoctorFarrHolz2017} as well as for a surrogate of eccentric BBH mergers~\cite{HuertaMooreKumar2017}. GPR has also been used in GW data analysis to marginalize over waveform uncertainties in parameter estimation~\cite{MooreGair2014, MooreBerryChua2016}.

This work is a continuation of the surrogate developed in Ref.~\cite{LackeyBernuzziGalley2017} for a nonspinning BNS EOB waveform using the description for the point-particle and tidal terms by Bernuzzi {\it et al}.~\cite{Bernuzzi:2014owa}. Whereas in Ref.~\cite{LackeyBernuzziGalley2017} we constructed a time-domain surrogate with the goal of reproducing the original waveform model as accurately as possible, in this work, we construct a frequency-domain model which significantly speeds up parameter estimation by not requiring an online Fourier transform for each waveform evaluation. \MP{Is it really true that this is a significant speedup? FFT may still be cheaper than surrogate waveform generation.} It also enables additional techniques to accelerate likelihood evaluations such as reduced order quadrature~\cite{Antil2013, CanizaresFieldGair2013, CanizaresFieldGair2015, Smith:2016qas} and multi-banding~\cite{VinciguerraVeitchMandel2017}. The resulting surrogate is the first EOB waveform model that consistently incorporates spin and tides and is fast enough for production parameter estimation codes.

We organize the paper as follows. In Section II we provide an overview of the \texttt{SEOBNRv4T} waveform model and the approximations used to reduce the dimensionality of the parameter space. In section III, we describe the details of building a surrogate model, choosing training set waveforms, and evaluating the final model. We then compare the accuracy and speed of the surrogate to the original model in section IV. Finally, we discuss applications and future improvements in Section V.

\textit{Conventions:} Unless explicitly stated, we use units where $G=c=1$.


%% ________________________________________________________
\section{Aligned-spin, dynamical tides EOB model}
\label{sec:eob}

The EOB approach to the general-relativistic 2-body problem, first described in Ref.~\cite{Buonanno:1998gg}, has proven successful in modeling the dynamics and GW emission of compact binaries. State-of-the-art aligned-spin EOB models~\cite{Bohe:2016gbl,Nagar:2017jdw} can accurately match hundreds of NR simulations of aligned-spin BBH systems for mass ratios up to 8 and spin magnitudes up to 0.85 for unequal-mass (up to 0.98 for equal-mass) binaries. The EOB framework can also accommodate precessing-spin BBHs, showing good agreement to mildly precessing NR simulations~\cite{Babak:2016tgq}. 

This research program has been extended to accommodate tidal effects~\cite{Damour:2009wj,Vines:2010ca,Damour:2012yf,Bini:2012gu,Bernuzzi:2014owa,Hinderer:2016eia,Steinhoff:2016rfi,Dietrich:2017feu} for binaries that contain NSs. The main effect of tides is to make the gravitational interaction more attractive with respect to the vacuum case. In particular, Refs.~\cite{Hinderer:2016eia,Steinhoff:2016rfi} built upon the aligned-spin BBH EOB model of Ref.~\cite{Taracchini:2013rva} and proposed a way to include the effect of dynamical tides. Neutron stars that are part of a compact-object binary will deform in the tidal field generated by the companion. The forcing tidal field varies at a multiple of the orbital frequency. Thus, in the late stages of the inspiral, the characteristic $f$-mode frequency of the neutron star can be dynamically approached, resulting in a resonant excitation of the $f$-mode. The net effect is an amplification of tidal effects as compared to the adiabatic limit, which assumes that the $f$-mode frequency is much larger than the frequency of the forcing tidal field. 

More precisely, spin effects are those of the point-mass model \texttt{SEOBNRv4}~\cite{Bohe:2016gbl} that was calibrated to 141 NR simulations of BBH systems with aligned spins. The model, however, does not account for the spin-induced quadrupole moment~\cite{Poisson1998}. This effect appears at 2PN order as compared to the tidal effects that first appear at 5PN order, but is only significant for systems with large spin. The importance of this physical effect in the context of data analysis for ground-based GW interferometers is currently being studied~\cite{HarryHinderer2017}.

\subsection{Inspiral-plunge waveform}

Dynamical tidal effects are implemented in the EOB model through a modification of the potential $\Delta_u$, which is the $tt$-component of the metric of the effective spacetime. We adopt the tidally-augmented expression for $\Delta_u$ discussed in Appendix~A of Ref.~\cite{Steinhoff:2016rfi}:
\begin{equation}
\Delta_u = \Delta_u^{\textrm{pm}} + \Delta_u^{\textrm{DT}}\,,
\end{equation}
where $\Delta_u^{\textrm{pm}}$ is the 4PN-accurate point-mass EOB term (Eq.~(2.2) of Ref.~\cite{Bohe:2016gbl}), and $\Delta_u^{\textrm{DT}}$ is the contribution due to dynamical tides. If either component of the binary is a black hole, then we set the tidal polarizabilities to zero. Let $M_{A,B}$ be the masses bodies $A$ and $B$, and $M = M_A + M_B$ be the total mass of the binary. The tidal contribution, including quadrupolar and octupolar dynamical tides, is
\begin{widetext}
\begin{align}
 \Delta_u^{\textrm{DT}} =&  - 3\,\Lambda_{2,\textrm{dyn}}^{A}(u)X_A^4 X_B\,u^6  \left[ 1 + \frac{5}{2} X_Au + \left(3 + \frac{1}{8}X_A + \frac{337}{28} X_A^2\right)u^2 \right]\nonumber\\
 &- 15\,\Lambda_{3,\textrm{dyn}}^{A}(u)X_A^6X_B\,u^8 \left[1 + \left(-2 + \frac{15}{2}X_A\right)u + \left(\frac{8}{3} - \frac{311}{24}X_A + \frac{110}{3}X_A^2\right) u^2\right] + (A \leftrightarrow B)\,.
\end{align}
\end{widetext}
Here, $X_{A,B}=M_{A,B}/M$, $u=1/r$ is the inverse of the ($M$-rescaled) EOB radial coordinate $r$, and $\Lambda_{\ell,\textrm{dyn}}^{A,B}(u)$ are the dimensionless $2^{\ell}$-polar dynamical tidal polarizabilities. Within the dynamical tides model, the tidal polarizabilities are not constant, but rather depend on the orbital separation and on the values of the $f$-mode frequencies $\hat{\omega}_{0\ell}^{A,B}$\footnote{Here, $\hat{\omega}_{0\ell}^{A,B}=M_{A,B}\omega_{0\ell}^{A,B}$, where $\omega_{0\ell}^{A,B}$ is the frequency in geometrized units.}. In particular, the dimensionless dynamical tidal polarizability reads
\begin{equation}
\Lambda_{\ell,\textrm{dyn}}^{A,B}(u)=\Lambda_{\ell}^{A,B}\hat{k}_{\ell\,\textrm{dyn}}(u;\hat{\omega}_{0\ell}^{A,B})\,,
\end{equation}
where $\Lambda_{\ell}^{A,B}$ is the dimensionless adiabatic tidal polarizability
\begin{equation}
\Lambda_{\ell}^{A,B}=\frac{2}{(2\ell-1)!!}\frac{k^{A,B}_{\ell}}{C_{A,B}^{2\ell+1}}\,,
\end{equation}
with $k^{A,B}_{\ell}$ the tidal Love number and $C_{A,B}=R_{A,B}/M_{A,B}$ the NS compactness, which depends on the NS radius $R_{A,B}$. Here, $\hat{k}_{\ell\,\textrm{dyn}}(u;\hat{\omega}_{0\ell}^{A,B})$ is the separation-dependent, dimensionless enhancement factor (Eq.~(11) of Ref.~\cite{Dietrich:2017feu}), which depends on the value of the $f$-mode $2^\ell$-pole dimensionless angular frequency, $\hat{\omega}_{0\ell}^{A,B}$. This enhancement factor is 1 for low orbital frequencies, and increases to $\sim 2$ as $\ell$ times the orbital frequency approaches the $f$-mode frequency. 

Tidal effects also enter the dissipative part of the model. In particular, the point-mass $\ell=2,3$ inspiral-plunge waveform modes are corrected by tidal terms
\begin{equation}
h_{\ell m}^{\textrm{insp-plunge}} = h_{\ell m}^{\textrm{pm}} + h_{\ell m}^{\textrm{tidal}}\,,\label{hlminspplunge}
\end{equation} 
where the point-mass piece $h_{\ell m}^{\textrm{pm}}$ is discussed in Section~II.B of Ref.~\cite{Bohe:2016gbl} and the tidal piece $h_{\ell m}^{\textrm{tidal}}$ is given by Eqs. (A14)-(A17) of Ref.~\cite{Damour:2012yf}. Following Ref.~\cite{Dietrich:2017feu} in the computation of $h_{22}^{\textrm{tidal}}$, we include a dynamical enhancement factor that depends on the orbital separation and on the $f$-mode frequency, and is given in Eq.~(15) of Ref.~\cite{Dietrich:2017feu}. The waveform modes $h_{\ell m}^{\textrm{insp-plunge}}$ are then used to calculate the gravitational-wave flux (Eqs.~(5) and~(6) of Ref.~\cite{Dietrich:2017feu}) from which the radiation-reaction force is derived. The Hamiltonian equations of motion with this radiation-reaction force are numerically integrated beginning with the quasicircular initial conditions described in Ref.~\cite{Buonanno:2005xu}. The inspiral-plunge waveform is obtained by evaluating Eq.~(\ref{hlminspplunge}) using the solution to the orbital dynamics. We only include the dominant $(\ell,m)=(2,2)$ mode in constructing the final waveform, although we use all modes when evaluating the radiation reaction force.

%Let $\Omega$ be the instantaneous orbital frequency, $r$ be orbital separation, $p_r$ be the radial momentum, $f_s$ be the waveform sampling frequency in Hz, $\omega_{\textrm{peak}}^{\textrm{NR}}$ be the NR fit for the (2,2)-mode amplitude peak of a BNS merger defined in Eq.~(2) of Ref.~\cite{Bernuzzi:2015rla}~\footnote{Note that this fit was tuned to NR simulations only for $0 \leq \kappa_2^T \leq 500$, where $\kappa_2^T$ is defined by Eq.~(1) of Ref.~\cite{Bernuzzi:2015rla}. In our model, for $\kappa_2^T >500$ we use the value of this fit at $\kappa_2^T=500$.}. Then the numerical integration of the orbital dynamics stops when any of the following events happens: (i) $\Omega$ peaks, (ii) $p_r$ becomes positive, (iii) $\dot{r}$ becomes positive, (iv) $\dot{p}_r$ becomes positive, (v) $r$ does not evolve more than a 1000th of the leading-order estimate for $\dot{r}$, that is $\propto 64/5 \nu/r^4$, (vi) $2\Omega$ reaches $\omega_{\textrm{peak}}^{\textrm{NR}}$, (vii) $2M\Omega$ reaches $\pi M f_s$. In order to avoid triggering these stopping conditions at low frequency, where residual orbital eccentricity due to imperfect quasicircular initial conditions could be present, we exploit the $\omega_{\textrm{peak}}^{\textrm{NR}}$ fit to define a radius below which the stopping conditions can be triggered; such threshold value is taken to be $1.5 (\omega_{\textrm{peak}}^{\textrm{NR}}/2)^{-2/3}$.  

We now look for a suitable time $t_{\rm match}$ to stop the numerical integration and match with a post-merger waveform. In the point-mass model \texttt{SEOBNRv4}, nonquasicircular corrections to the inspiral-plunge signal guarantee that the waveform peaks, and $t_{\rm match}$ is chosen to be the time of the peak amplitude $t_{\rm peak}^{\rm amp}$. Here, we do not use nonquasicircular corrections tuned to BNS simulations \red{[[Ambiguous. Do we use them or not. Find out.]]}, so we have less control over the behavior of Eq.~(\ref{hlminspplunge}) in the late inspiral. We therefore choose the following definition for $t_{\rm match}$. Let $t_{\rm peak}^{\rm amp}$ be the earliest time when the amplitude $|h_{22}^{\textrm{insp-plunge}}|$ of the (2, 2) mode peaks, and let $t_{\rm peak}^{\rm freq}$ be the time when the frequency $\omega_{22}^{\rm insp-plunge}$ of the (2, 2) mode peaks. Then,
\begin{equation}
t_{\textrm{match}} = \min \left(t_{\textrm{peak}}^{\textrm{amp}},t_{\textrm{peak}}^{\textrm{freq}}\right)\,.\label{tmatch}
\end{equation}
For some waveforms, the amplitude does not peak. In this case, we choose $t_{\rm peak}^{\rm amp}$ to be the earliest time when the slope of the amplitude $\partial_t |h_{22}^{\rm insp-plunge}|$ reaches a minimum after having reached a peak.

%In the BBH model, nonquasicircular corrections to the inspiral-plunge signal guarantee that the waveform peaks at a prescribed time and has prescribed amplitude and frequency, hence $t_{\textrm{match}}^{22}=t_{\textrm{peak}}^{\textrm{amp}}$. Here, however, we do not impose such corrections because of limited input from NR BNS simulations, therefore we have little control over the behavior of Eq.~(\ref{hlminspplunge}) in the late inspiral. This motivates our choice of $t_{\textrm{match}}^{22}$ in Eq.~(\ref{tmatch}), with the goal of reducing to the BBH prescription in the non-tidal limit and avoiding non-monotonic growth of the GW frequency in the late inspiral. To help bracket the stopping time, we exploit a fit for the (2,2)-mode peak frequency of NR BNS simulations~\cite{Bernuzzi:2015rla} to estimate the final orbital frequency of the EOB inspiral-plunge dynamics. 

%Ref.~\cite{Dietrich:2017feu} carried out comprehensive comparisons of this model to state-of-the-art NR simulations of coalescing nonspinning BNSs with different mass ratios and equations of state. The model can accurately describe the GW phasing up to the last cycle before merger for systems with soft equation of state, while improvements are still needed for systems with stiff equation of state. \red{[[To what accuracy? Is this paragraph already in the intro and can be deleted?]]} 


\subsection{Post-merger waveform}

In BNS simulations, the waveform rapidly tapers to zero amplitude in 1--2 gravitational-wave cycles after reaching peak amplitude. The BNS system then either undergoes prompt collapse or post-merger oscillations of the remnant. We model this tapering in terms of analytic functions for the amplitude $A^{\textrm{post-mrg}}(t)$ and phase $\phi^{\textrm{post-mrg}}(t)$ such that $h_{22}^{\textrm{post-mrg}}(t) = A^{\textrm{post-mrg}}(t) \exp{[i\phi^{\textrm{post-mrg}}(t)]}$. 

For the amplitude, we smoothly extend the waveform after $t_{\rm match}$ with a linear fit, and then taper the resulting amplitude. The linear extension is defined by
\begin{align}
\hat{A}(t)=\left\{\begin{array}{ll}
|h_{22}^{\rm insp-plunge}(t)|, & t \leq t_{\rm match}\,, \\
a + b (t - t_{\rm match}), & t > t_{\rm match}\,,
\end{array}\right.
\end{align}
where $a = |h_{22}^{\rm insp-plunge}(t_{\rm match})|$ and $b=\partial_t |h_{22}^{\rm insp-plunge}(t_{\rm match})|$. The tapering function is centered $15M$ after $t_{\rm match}$ and has a decay time $\tau = \pi/(2\omega_{\rm match})$ (where $\omega_{\rm match} = \omega_{22}^{\rm insp-plunge}(t_{\rm match})$) of one gravitational wave period. \red{[[As written, this is a decay time of 1/4 of a period. Figure out what is correct.]]} It is given by
\begin{equation}
W(t) = \frac{1}{1+\exp{[(t- t^{22}_{\textrm{match}}-15M)/\tau]}}\,.
\end{equation}
The final amplitude after windowing is then given by $|h_{22}(t)| = \hat A(t) W(t)$.

For the phase, we smoothly extend the waveform frequency such that it agrees with the inspiral frequency at a time $t_{\rm freq} = t_{\rm match} - 20M$ before the matching time $t_{\rm match}$, but then stretches out the frequency evolution such that it only approaches $\omega_{\rm match}$ asymptotically. We define this frequency evolution as
\begin{align}
\omega(t) = \omega_{\rm match} - \Delta\omega \exp\left[ -(t - t_{\rm freq}) / (20M) \right],
\end{align}
where $\Delta\omega = \omega_{\rm match} - \omega_{\rm freq}$. Integrating and requiring continuity at $t_{\rm freq}$ results in the final expression for the phase
\begin{align}
\phi_{22}(t) = \left\{\begin{array}{ll}
\phi_{22}^{\textrm{insp-plunge}}(t), & t \leq t_{\rm freq} \,, \\
\phi_{\rm freq} + \omega_{\rm match}(t - t_{\rm freq}) + 20M \Delta\omega \left\{ \right. & \\
  \left. \exp\left[ -(t - t_{\rm freq}) / (20M) \right] - 1\right\} \,,& t > t_{\rm freq} \, .
\end{array}\right.
\end{align}

Although this post-merger model has not been fit to NR BNS simulations, it is in reasonable qualitative agreement with the post-merger behavior of the two equal-mass nonspinning NR BNS simulations that were analyzed in Ref.~\cite{Hinderer:2016eia}. A more sophisticated, NR-informed model of the post-merger emission will be part of future investigations.


\subsection{Approximation of higher-order tidal parameters}

After rescaling with the total mass $M$, an aligned-spin EOB waveform with dynamical tides depends on 11 intrinsic parameters: the mass ratio $q=M_B/M_A\leq1$, the two dimensionless components of the spins along the Newtonian orbital angular momentum $-1 \leq S_{A,B} \leq 1$, the two adiabatic quadrupolar tidal polarizabilities $\Lambda_2^{A,B}$, the two adiabatic octupolar tidal polarizabilities $\Lambda_3^{A,B}$, and the four $\ell=2,3$ fundamental $f$-mode angular frequencies $\omega_{0\ell}^{A,B}$. To reduce the dimensionality of the intrinsic parameter space, we exploit nearly EOS-independent fits for the higher-order effects in terms of $\Lambda_2$ usually referred to as universal relations. In particular, Eq.~(60) of Ref.~\cite{Yagi:2013sva} provides a fit for $\Lambda_3$ as a function of $\Lambda_2$ with an accuracy of about 10\% depending on the EOS. This relation has been fit for a sample of EOSs and for tidal parameters in the range $1 \leq \Lambda_2\leq10^6$. This formula, however, diverges at small values of $\Lambda_2$, so we replace it with a polynomial function that vanishes at $\Lambda_2 = 0$. We require this extension to be continuous at $\Lambda_2 = 10^{-2}$, and to fit the universal relation in the range $10^{-5}\leq\Lambda_2\leq10^{-2}$. The exact relation we use is
\begin{equation}
\Lambda_3 = \left\{\begin{array}{ll}
\Lambda_2 (f + g \Lambda_2 + h \Lambda_2^2)\, , & 0 \leq \Lambda_2 \leq 10^{-2} \\
e^{a + b \xi + c \xi^2 + d \xi^3 + e \xi^4}\, , & \Lambda_2>10^{-2}
\end{array}\right.,
\end{equation}
where $\xi = \ln\Lambda_2$, $\{a, b, c, d, e\} = \{-1.15, 1.18, 2.51\times 10^{-2}, -1.31\times 10^{-3}, 2.52\times 10^{-5}\}$ from Ref.~\cite{Yagi:2013sva}, and $\{f, g, h\}=\{0.440649, -34.632322, 1762.112913\}$.

Eq. (3.5) of Ref.~\cite{Chan:2014kua} gives relations for $\omega_{02}$ as a function of $\Lambda_2$ and $\omega_{03}$ as a function of $\Lambda_3$ to within a few percent error. For $\omega_{02}$, the fitting range used by Ref.~\cite{Chan:2014kua} was $0\leq \xi \leq 9$, and outside this range we require continuity. The relation we use is then
\begin{equation}
\omega_{02} = \left\{\begin{array}{ll}
f\, , & \xi < 0 \\
a + b\xi + c\xi^2 + d\xi^3 + e\xi^4\, , & 1 \le \xi \le 9 \\
g\, , & \xi > 9
\end{array}\right.,
\end{equation}
where $\{a, b, c, d, e\} = \{0.182, -6.836\times10^{-3}, -4.196\times10^{-3}, 5.215\times10^{-4}, -1.857\times10^{-5}\}$ from Ref.~\cite{Chan:2014kua}, and $\{f, g\}=\{0.182, 0.161\}$. For $\omega_{03}$, the fit is given in terms of $\Upsilon=\ln\Lambda_3$ in the range $-1\leq \Upsilon \leq 10$, and outside this range we require continuity. The relation we use is then
\begin{equation}
\omega_{03} = \left\{\begin{array}{ll}
f\, , & \Upsilon < -1 \\
a + b\Upsilon + c\Upsilon^2 + d\Upsilon^3 + e\Upsilon^4\, , & -1 \le \Upsilon \le 10 \\
g\, , & \Upsilon > 10
\end{array}\right.,
\end{equation}
where $\{a, b, c, d, e\} = \{0.2245, -1.5\times10^{-2}, -1.412\times10^{-3}, 1.832\times10^{-4}, -5.561\times10^{-6}\}$ and $\{f, g\}=\{0.2379, 0.1165\}$.

With these relations the waveform only depends on 5 intrinsic parameters $\bx=\{q, \chi_A, \chi_B, \Lambda_2^A,\Lambda_2^B\}$. The aligned component spins are defined as $\chi_i = \hat L_N \cdot \vec S_i / m_i^2$ where $\vec S_i$ are the dimensionfull spin vectors, $\hat L_N$ the Newtonian orbital angular momentum and $m_i$ the component masses.
We note that this list does not include the total mass $M$ of the system. The point-mass part of the model is scale-invariant, and the tidal corrections only depend on $X_A=1/(1+q)$, $X_B=q/(1+q)$ and $\Lambda_2^{A,B}$. Finally, the simple model of the post-merger signal that we employ rescales with $M$ as well. It is not clear at what point in the transition from the inspiral to the post-merger this approximation breaks down. For example, whether the merging binary undergoes prompt collapse or forms a hypermassive remnant depends sensitively on the total mass. A set of NR simulations with a fixed mass ratio and tidal parameters but different total masses (and therefore different EOSs in order to keep the tidal parameters fixed) will be needed to determine the time when this approximation breaks down.

\red{State range of validity for this model.}

\red{State that you will use the notation $\Lambda_1$, $\Lambda_2$ for the remainder of the paper.}

%% ________________________________________________________
\section{Surrogate model}

In this section we describe how we decompose the EOB waveform into smooth, slowly-varying functions and train a surrogate model for these functions. We will work with the frequency-domain waveform $\tilde h(Mf; \bx)$. This is favorable for data analysis which is usually done in the frequency domain. It also allows us to extend the model down to arbitrarily low frequencies with analytic, frequency-domain PN approximations. Unfortunately, Fourier transforming a finite length, discretely sampled waveform leads to a surrogate with more noise. We will show below, however, that sufficient filtering can solve this problem.


\subsection{Decomposition of the waveform}

Because the waveform $\tilde h(Mf; \bx)$ is an oscillatory function of $Mf$ and $\bx$, the waveform is usually decomposed into an amplitude $A(Mf; \bx)$ and phase $\Phi(Mf; \bx)$ as $\tilde h(Mf; \bx) = A(Mf; \bx) e^{i\Phi(Mf; \bx)}$. The amplitude and phase are smoother, mostly monotonic functions of frequency \MP{the phase is not guaranteed to be monotonic}. This can be seen in Fig.~\ref{fig:hoff} where we show the waveforms for the 32 corners of the 5-dimensional parameter space. Unfortunately, the amplitude and phase still span a wide range of values. The phase, for example, spans about $10^4$~rad between waveforms with different parameters $\bx$ (see Fig.~\ref{fig:hoff}). To avoid systematic errors in the tidal parameters, we need phase errors of $\lesssim 1$~rad over most of this frequency range, leading to a requirement on the fractional interpolation error of $\lesssim 10^{-4}$~rad. This is a difficult requirement to achieve for 5-dimensional interpolation. 

\begin{figure}[htb]
\centering
\includegraphics[width=0.49\textwidth]{hoff.pdf}
\caption{EOB waveforms $\tilde h(Mf; \bx)$ for the 32 corners of parameter space. The waveforms are filtered and Fourier transformed as described in Sec.~\ref{sec:condition}. Top two panels: the amplitude $A(Mf; \bx)$ and residual $\Delta\ln A(Mf; \bx)$ relative to TaylorF2 as defined in Eq.~\eqref{eq:ampres}. Bottom two panels: the phase $\Phi(Mf; \bx)$ and residual $\Delta\Phi(Mf; \bx)$ relative to TaylorF2 as defined in Eq.~\eqref{eq:phaseres}. Vertical dashed lines represent the frequency nodes $MF_j$ where the residuals are interpolated as functions of $\bx$ using Gaussian process regression. The vertical solid line at $Mf_{\rm ISCO} = 1/(6^{3/2}\pi) \approx 0.022$ is the gravitational-wave frequency at the Schwarzschild ISCO.}
\label{fig:hoff}
\end{figure}

For aligned-spin waveforms, we can solve this problem by using the fact that the waveform can be approximated with the analytic \texttt{TaylorF2} waveform $\tilde h_{\rm F2}(Mf; \bx) = A_{\rm F2}(Mf; \bx) e^{i\Phi_{\rm F2} (Mf; \bx)}$. This allows us to write the EOB amplitude and phase in terms of small residuals, $\Delta\ln(A)(Mf; \bx)$ and $\Delta\Phi(Mf; \bx)$, relative to TaylorF2
\begin{align}
\Delta\ln(A)(Mf; \bx) &= \ln\left(\frac{ A(Mf; \bx) }{ A_{\rm F2}(Mf; \bx)}\right), \label{eq:ampres}\\
\Delta\Phi(Mf; \bx) & = \Phi(Mf; \bx) - \Phi_{\rm F2}(Mf; \bx), \label{eq:phaseres}
\end{align}
such that
\begin{align}
\tilde h(Mf; \bx) = \tilde h_{\rm F2}(Mf; \bx) e^{\Delta\ln(A)(Mf; \bx) + i  \Delta\Phi(Mf; \bx)}.
\label{eq:hdecomp}
\end{align}
These residuals are also shown in Fig.~\ref{fig:hoff}. For the amplitude residual, we use a log-ratio instead of a ratio because it guarantees that interpolation errors will not lead to a negative amplitude for the reconstructed waveform (Eq.~\eqref{eq:hdecomp}). In addition, because the waveform amplitude spans several orders of magnitude after the merger frequency, the log-ratio captures this behavior. Comparing the phase $\Phi$ and phase residual $\Delta\Phi$ in Fig.~\ref{fig:hoff}, we find that the range in $\Delta\Phi$ is a few orders of magnitude smaller than the range in $\Phi$ except at very high frequencies. In Fig.~\ref{fig:dhofs}, we show the amplitude and phase residuals at three fixed frequencies as functions of the waveform parameter $\chi_1$. These quantities are smooth functions of $\bx$, implying that interpolation as a function of $\bx$ should work. The exception is the amplitude residual at very high frequencies where the amplitude is small and numerical noise is significant.

\begin{figure}[htb]
\centering
\includegraphics[width=0.49\textwidth]{dhofs.pdf}
\caption{Cross sections of the amplitude and phase residuals as functions of $\chi_1$ at three nodes $MF_j$. The waveforms are generated for 51 values of $\chi_1$ shown as dots with the other parameters held constant at $\{q, S_2, \Lambda_1, \Lambda_2\} = \{0.6, 0.2, 2000, 1000\}$. Also shown are 1-dimensional cross sections of the interpolated residuals using Gaussian process regression (solid black) as well as their estimated $1\sigma$ uncertainty (shaded gray region).}
\label{fig:dhofs}
\end{figure}

The explicit expressions for the amplitude and phase of the TaylorF2 waveform are as follows. For the amplitude we use the 1PN correction to the leading order waveform.
\begin{equation}
A_{\rm F2} = \sqrt{\frac{5\pi\eta}{24}} x^{-7/4} \left[ 1 + \left(-\frac{323}{224} + \frac{451\eta}{168} \right)x \right],
\end{equation}
where $x=(\pi M f)^{2/3}$ is the standard PN parameter. This differs slightly from the decomposition used by \texttt{LAL}~\cite{lal}. The phase has the schematic form
\begin{align}
\Phi_{\rm F2} =& -2\pi f t_c + \phi_c + \frac{\pi}{4} - \frac{3}{128\eta}x^{-5/2} \left[1 + \right. \\
                        & \left. {\rm PP}(\eta) + {\rm Spin}(\eta, \chi_1, \chi_2) + {\rm Tidal}(\eta, \Lambda_1, \Lambda_2)  \right].
\end{align}
We use the point particle terms ${\rm PP}(\eta)$ to 3.5PN order in Eq.~(3.18) of Ref.~\cite{BuonannoIyerOchsner2009}, the aligned spin terms ${\rm Spin}(\eta, \chi_1, \chi_2)$ to 3PN order~\cite{BoheMarsatBlanchet2013}, and the tidal terms ${\rm Tidal}(\eta, \Lambda_1, \Lambda_2)$ to 6PN order~\cite{VinesFlanaganHinderer2011}. These terms for the phase are exactly as used in the \texttt{LAL} waveform \texttt{TaylorF2}. \MP{add git hash? LAL's TF2 will change in the future when new terms become available.}


\subsection{Conditioning the training set waveforms}
\label{sec:condition}

The accuracy of the final frequency-domain surrogate depends on how well the finite-length, numerical waveform is Fourier transformed and filtered to remove numerical artifacts. We now describe the procedure to condition the training-set waveforms used to construct the surrogate.

We evaluate the EOB waveform with a starting frequency of $Mf_{{\rm win}, i}=0.000197$, equivalent to a physical frequency of 20~Hz for a binary with total mass $M=2M_\odot$. In order to take a discrete Fourier transform, we window the start of the waveform with a Planck window~\cite{McKechanRobinsonSathyaprakash2010} in the interval $[Mf_{{\rm win}, i}, Mf_{{\rm win}, f}] = [0.000197, 0.00021]$ to avoid Gibbs oscillations. The end of the waveform has zero amplitude, so the end does not need to be windowed. We then resample the waveform with a spacing $\Delta t/M = 5$ and pad the end of the waveform with zeros such that all waveforms in the training set have the exact same time samples. After evaluating the discrete Fourier transform, we calculate the residuals $\Delta\ln(A)(Mf)$ and $\Delta\Phi(Mf)$ between the EOB and TaylorF2 waveforms using Eqs.~\eqref{eq:ampres} and~\eqref{eq:phaseres}. 

Waveforms have free time and phase parameters $t_c$ and $\phi_c$, and in the frequency-domain this means that one can freely add a linear term $\phi_c + 2\pi (Mf) (t_c/M)$ to the phase $\Phi(Mf)$. We use this freedom to match the EOB waveform to the analytic TaylorF2 waveform near the starting frequency. We do this by subtracting a linear fit to $\Delta\Phi(Mf)$ at the beginning of the waveform in the window $[Mf_{{\rm fit}, i}, Mf_{{\rm fit}, f}] = 0.00021[1, 1.05]$. At $Mf=Mf_{{\rm fit}, i}$, the resulting phase residual $\Delta\Phi(Mf)$ is zero and has zero slope, guaranteeing that the surrogate smoothly matches to TaylorF2 below this frequency.

The resulting waveforms still have some remaining Gibbs oscillations which can be seen as small-amplitude, high-frequency oscillations in the residuals $\Delta\ln(A)(Mf)$ and $\Delta\Phi(Mf)$. These come from two sources. The first is the fact that the Planck window at the beginning of the waveform was not sufficiently long. We could make this window longer, but that would require us to start the surrogate at a higher frequency. The second source comes from the end of the waveform where the amplitude rapidly drops to zero amplitude after the peak amplitude in $\sim 1$ cycle. The Gibbs oscillations at high frequencies are therefore a genuine feature of the EOB model. In an attempt to preserve all structure in the EOB waveform, we choose not to window the end of the time-domain waveform. We instead remove these oscillations in the frequency domain using a moving average filter centered on $Mf$ with width $[Mf(1-\Delta Mf), Mf(1+\Delta Mf)]$. We use a width of $\Delta Mf=0.1$ for the amplitude residual and $\Delta Mf=0.05$ for the phase residual. Smoothing these oscillations makes it significantly easier to interpolate the amplitude and phase residuals as functions of $Mf$ and $\bx$. \red{[[$\Delta MF$ is a difference not a fraction. maybe try $p$ or something similar? Mention that 0.1 for the amplitude is a bit aggressive, but only matters for high frequency where the amplitude is crazy anyway.]]} 

Finally, we truncate the residuals outside the interval $[Mf_{{\rm trunc}, i}, Mf_{{\rm trunc}, f}] = [0.00021, 0.07]$. We note that the gravitational-wave frequency at the Schwarzschild ISCO is $Mf_{\rm ISCO} = 1/(6^{3/2}\pi) \approx 0.022$. Although second generation detectors are only marginally sensitive to BNS signals above this frequency, we use the high-frequency cutoff of $Mf_{{\rm trunc}, f}=0.07$ because the amplitude of some of the EOB waveforms is still large enough at these high frequencies to affect the structure of the surrogate waveform if inverse Fourier transformed back into the time domain. \MP{Say that Mf=0.07 is a (generous?) upper bound on the NR BNS merger frequency and thus a safe choice.}

To validate that the waveforms are sufficiently conditioned, we plot cross-sections of the residuals in Fig.~\ref{fig:dhofs} for fixed frequencies as functions of one of the waveform parameters $\chi_1$. The fact that the residuals are smooth functions of the waveform parameters indicates that most numerical noise has been removed. The exception is the noisy amplitude residual at high frequencies, $\Delta\ln(A)(Mf=0.052; \bx)$, where the amplitude is very small and dominated by numerical noise. 
\MP{The discussion of Fig.~\ref{fig:dhofs} appears twice in the text; here and one section earlier.}

\subsection{Spline interpolation for frequency $f$}

We now seek to interpolate the conditioned residuals $\Delta\ln(A)(Mf; \bx)$ and $\Delta\Phi(Mf; \bx)$. We begin by choosing a method for interpolating as a function of $Mf$ for fixed $\bx$. The majority of previous papers on gravitational-wave surrogates have used an orthonormal basis of global functions $\hat e_i(Mf)$ for interpolating as a function of frequency (or time)~\cite{Puerrer2014, Puerrer2015, FieldGalleyHesthaven2014, LackeyBernuzziGalley2017, BlackmanFieldGalley2015, BlackmanFieldScheel2017a, BlackmanFieldScheel2017b}. This type of surrogate, built from a reduced basis, is usually referred to as a reduced order model. For a generic function $g(Mf; \bx)$, this decomposition can be written as $g(Mf; \bx) \approx \sum_{i=1}^N c_i(\bx) \hat e_i(Mf)$. The coefficients $c_i(\bx)$ are then interpolated as functions of $\bx$~\cite{Puerrer2014, Puerrer2015}. Alternatively, using the empirical interpolation method~\cite{Barrault2004, Chaturantabut2010, FieldGalleyHesthaven2014}, one can re-express these basis functions $\hat e_i(Mf)$ in terms of empirical interpolating functions $B_j(Mf)$ and the value of $g(Mf; \bx)$ at empirical nodes $MF_j$ as $g(Mf; \bx) \approx \sum_{j=1}^N B_j(Mf) g(MF_j; \bx)$. The location of these nodes $MF_j$ is then optimized to minimize interpolation errors.

For the problem here, we have found that global basis functions do not work well. \MP{EIM did not work well; is this also true for the SVD basis and projection coefficient method?} The residuals are small and smooth at low frequencies and large and noisy at high frequencies. With global basis functions, errors in evaluating $g(MF_j;\bx)$ at high frequencies can propagate to large errors between the nodes $MF_j$ at low frequencies. Instead, we find that spline interpolation works significantly better. We use 40 frequency nodes $MF_j$ log-spaced in the interval $[Mf_{{\rm trunc}, i}, Mf_{{\rm trunc}, f}]$ (see Fig.~\ref{fig:hoff}). We evaluate the residuals $\Delta\ln(A)(MF_j; \bx)$ and $\Delta\Phi(MF_j; \bx)$ at these nodes as discussed below. We then interpolate between these frequencies using third-order splines. The local, third-order polynomials, that are only connected by the requirement of smoothness, do not propagate high-frequency errors down to low-frequency errors as significantly as do global basis functions.


\subsection{Gaussian process regression for parameters $\bx$}

Next we choose a method to interpolate the residuals $\Delta\ln(A)(MF_j; \bx)$ and $\Delta\Phi(MF_j; \bx)$ at each of the frequency nodes $MF_j$ as a function of the five waveform parameters $\bx$. Most multivariate interpolation techniques (e.g. tensor spline or Chebyshev interpolation) require a function to be sampled on a rectangular grid, and thus suffer from the curse of dimensionality: the number of samples grows exponentially with the dimension $d$ ($N^d$ samples for $N$ points per dimension). For our 5-dimensional problem, $10^5$ waveform evaluations are needed for only 10 samples per parameter. At $\sim10$~minutes--1~day per EOB waveform on a standard CPU \MP{For which $f_\text{min}$ are these times given?}, this is at the limit of what is reasonable. However, there are other methods that do not require a rectangular grid, and this allows us to choose more efficient experimental designs as discussed in Sec.~\ref{sec:design} below. The method we choose is Gaussian Process Regression (GPR)~\cite{RasmussenWilliams2006}.

In GPR a function $g(\bx)$ is described in terms of a mean $m(\bx)$ and  covariance $k(\bx, \bx')$ between points.
\begin{equation}
g(\bx) \approx \mathcal{GP}(m(\bx), k(\bx, \bx')).
\end{equation}
The mean $m(\bx)$ can be a parameterized fitting function while $k(\bx, \bx')$ is a kernel with tunable hyperparameters that describes the covariance between the point $\bx$ and the sampled point $\bx'$. This covariance can be used to describe features of the function not captured by the parameterized mean $m(\bx)$ as well as genuine noise in the data $\by$. For the problem here, we have already subtracted the TaylorF2 waveform from the EOB waveform, so we set $m(\bx) = 0$ and model the residuals purely in terms of the kernel $k(\bx, \bx')$. 

In a zero-mean Gaussian process, the data $y_i$ and function value $y_*$ at the training set points $\bx_i$ and new point $\bx_*$, respectively, are drawn from a multivariate normal distribution
\begin{equation}
\label{eq:gaussian}
\begin{bmatrix}
{\bm y} \\
y_* \\
\end{bmatrix}
\sim \mathcal{N}
\left({\bm 0}, 
\begin{bmatrix}
K & K_*^T \\
K_* & K_{**} \\
\end{bmatrix}
\right)
\end{equation}
where $K_{ij} = k(\bx_i, \bx_j)$ is a matrix, $K_{*i} = k(\bx_i, \bx_*)$ is a vector, and $K_{**} = k(\bx_*, \bx_*)$ is a scalar.

The conditional probability for $y_*$ given the training set examples $\by$ and kernel hyperparameters $\btheta$ is also a Gaussian
\begin{equation}
p(y_* | \bx_i, \bx_*, \by, \btheta) \sim \mathcal{N}(\bar y_*, {\rm Var}(y_*)),
\end{equation}
where the mean and variance are
\begin{align}
\label{eq:mean}
\bar y_* &= K_*^T K^{-1} y, \\
\label{eq:var}
{\rm Var}(y_*) &= K_{**} - K_*^T K^{-1} K_*.
\end{align}
Eq.~\eqref{eq:mean} is the estimate of the function, and Eq.~\eqref{eq:var} is the estimate of the uncertainty. We note that the variance does not depend on the data $\by$.

We use a radial kernel which expresses the covariance in terms of a distance $r$ between points
\begin{equation}
r^2 = (\bx - \bx')^T M (\bx - \bx'),
\end{equation}
where we choose the matrix $M$ to be diagonal
\begin{equation}
M = {\rm diag}(\ell_1^{-2}, \ell_2^{-2}, \dots, \ell_d^{-2}).
\end{equation}
The tunable hyperparameters $\ell_i$ represent the length scale over which the function $g(\bx)$ varies in each coordinate $x_i$. Specifically, we use the Mat\'{e}rn class of kernels
\begin{equation}
k_{\rm Matern}(r) = \frac{2^{1-\nu}}{\Gamma(\nu)} \left(\sqrt{2\nu} r\right)^\nu K_\nu \left(\sqrt{2\nu} r \right).
\end{equation}
where $K_\nu(x)$ is a modified Bessel function. The value of $\nu$ parameterizes the smoothness of the Gaussian process, and is $k$ times mean-square differentiable if $\nu>k$~\cite{RasmussenWilliams2006}. For half-integer values of $\nu$, this kernel has a computationally cheap form without special functions, and we have had good results with $\nu=5/2$, resulting in a twice-differentiable function\footnote{If desired, one can select the optimal class of kernels using cross validation~\cite{RasmussenWilliams2006}, and we have also tried the more common, infinitely differentiable squared exponential kernel~\cite{RasmussenWilliams2006}. However, we found it difficult to reliably tune its hyperparameters for the problem here.}. The $\nu=5/2$ kernel is
\begin{equation}
k_{\rm Matern}^{\nu=5/2}(r) = \left(1+\sqrt{5}r + \frac{5r^2}{3}\right) \exp\left(-\sqrt{5}r\right).
\end{equation}

Our final kernel takes the form
\begin{equation}
k(\bx_i, \bx_j) = \sigma_f^2 k_{\rm Matern}^{\nu=5/2}(r) + \sigma_n^2 \delta_{ij}
\end{equation}
where $\sigma_f$ is a scale factor that describes the range of values that $g(\bx)$ takes over the domain, and $\sigma_n$ is a noise parameter. The white noise kernel $\sigma_n^2 \delta_{ij}$ (also called a nugget) is $\sigma_n^2$ when $\bx_i = \bx_j$ and zero otherwise, and parameterizes noise in the data $\by$. In our case, the training set waveforms have numerical noise that we will estimate by optimizing the hyperparameters. The full set of hyperparameters is now  $\btheta = \{\sigma_f, \ell_q, \ell_{\chi_1}, \ell_{\chi_2}, \ell_{\Lambda_1}, \ell_{\Lambda_2}, \sigma_n\}$.

In order to estimate the hyperparameters, we use the above assumption (Eq.~\eqref{eq:gaussian}) that the joint distribution of 
the data ${\bm y}$ is a multivariate Gaussian which has the following distribution
\begin{equation}
\ln p({\bm y} | {\bm x}, {\bm \theta}) = -\frac{1}{2}{\bm y}^T K^{-1} {\bm y} - \frac{1}{2} \ln |K| - \frac{d}{2} \ln 2\pi.
\end{equation}
This is the log-likelihood for ${\bm y}$ given the hyperparameters ${\bm \theta}$, and we can find the posterior for ${\bm \theta}$
given ${\bm y}$ using Bayes' theorem
\begin{equation}
p({\bm \theta} | {\bm x}, {\bm y}) \propto p({\bm \theta}) p({\bm y} | {\bm x}, {\bm \theta}).
\end{equation}
The prior $p({\bm \theta})$ is typically uniform and used to set the bounds on ${\bm \theta}$. One can sample this posterior if interested in the distribution of hyperparameters $\btheta$. However, for the problem here, we simply want the maximum posterior. We do this using the \texttt{gaussian\_process} module in the \texttt{scikit-learn} package~\cite{scikit-learn}. With these optimized hyperparameters, the final interpolating function is given by Eq.~\eqref{eq:mean} and its uncertainty by Eq.~\eqref{eq:var}. We note that once the vector $K^{-1}y$ is precomputed, evaluating Eq.~\eqref{eq:mean} is a fast $\mathcal{O}(N)$ operation.


\subsection{Surrogate waveform evaluation}

\MP{Say that we use sparse frequency grids based on power-law $\Delta f$ in the evaluation of the surrogate; perhaps describe in more detail how the model is evaluated in the LAL code.}
\MP{Discuss linear vs cubic splines if we decide to use them. Tradeoff in accuracy and speed.}

Given the interpolating functions for the amplitude and phase residuals $\{\mathcal{I}_{\rm GPR}[\Delta\ln A_j](\bx)\}$ and $\{\mathcal{I}_{\rm GPR, j}[\Delta\Phi_j](\bx)\}$ evaluated at the frequency nodes $MF_j$, we can now reconstruct the frequency-domain waveform. The surrogates for the residuals $\Delta\ln A_S(Mf;\bx)$ and $\Delta\Phi_S(Mf;\bx)$ are constructed by interpolating between the nodes $MF_j$ with cubic splines:
\begin{align}
\Delta\ln A_S(Mf;\bx) &= \mathcal{I}_{\rm Spline}[\{\mathcal{I}_{\rm GPR}[\Delta\ln A_j](\bx)\}](Mf),\\
\Delta\Phi_S(Mf;\bx) &= \mathcal{I}_{\rm Spline}[\mathcal{I}_{\rm GPR}[\{\Delta\Phi_j](\bx)\}](Mf).
\end{align}
We set these functions to 0 below the first frequency node $MF_0 =  Mf_{trunc,i}$ so that the waveform transitions to TaylorF2 at lower frequencies. With the analytic expressions for $A_{\rm F2}(Mf;\bx)$ and $\Phi_{\rm F2}(Mf;\bx)$ and the interpolated expressions $\Delta\ln A_S(Mf;\bx)$ and $\Delta\Phi_S(Mf;\bx)$, the final surrogates for the amplitude and phase are
\begin{align}
A_S(Mf;\bx) &= A_{\rm F2}(Mf;\bx) \exp\left[ \Delta\ln A_S(Mf;\bx) \right] \\
\Phi_S(Mf;\bx) &= \Phi_{\rm F2}(Mf;\bx) + \Delta\Phi_S(Mf;\bx).
\end{align}
In physical units, for an inclination angle $\iota$, the $+$ and $\times$ polarizations of the waveform are 
\begin{align}
\tilde h_+(f; \bx) &= \frac{1}{2}(1+\cos^2\iota) \frac{G^2 M^2}{c^5 d} A_S\left(\frac{GMf}{c^3}; \bx\right) \nonumber \\
& \times \exp\left[i \Phi_S\left(\frac{GMf}{c^3}; \bx\right)\right], \\
\tilde h_\times(f; \bx) &= \cos\iota \frac{G^2 M^2}{c^5 d} A_S\left(\frac{GMf}{c^3}; \bx\right) \nonumber \\
& \times \exp\left[i \Phi_S\left(\frac{GMf}{c^3}; \bx\right) + i \frac{\pi}{2}\right].
\end{align}


\subsection{Iterative construction of training set and surrogate}
\label{sec:design}

One of the main aims of this paper is to build a surrogate with as few waveform evaluations as possible. Using the freedom provided by GPR to sample waveforms at arbitrary locations, we try a set of designs more efficient than uniform grids. One such design is a Latin Hypercube Design (LHD)~\cite{McKayBeckmanConover1979}. An LHD with $N$ samples divides each of $d$ dimensions uniformly into $N$ grid points for a total of $N^d$ grid points. However, unlike a uniform grid, the $N$ values in each dimension are sampled exactly once instead of $N^{d-1}$ times. For an LHD there are $(N!)^d$ ways to choose these points, and we choose one randomly\footnote{There are ways to better choose an LHD. A standard requirement is that the LHD be space-filling, meaning that the points are as far apart as possible from each other. (See Ref.~\cite{Husslage2011} for a review of methods for optimizing the placement of samples.) One such definition of space filling is that the chosen locations maximize the minimum Euclidean distance between any two samples. We have not experimented with these alternatives here.}. 

An LHD has the property that the samples are non-collapsing; a projection onto a subspace is still an LHD and no points are repeated in any dimension. This avoids wasting samples when one of the parameters has much less of an influence than the other parameters. As an example, the tidal parameters have a minimal impact on the waveform at low frequencies, so it would not make sense to densely sample a grid of tidal parameters at low frequencies. However, during the merger the tidal parameters can be more important than the other parameters. One therefore might want more samples for the tidal parameters and less samples for the other parameters. However, due to the expense of the waveforms we must use the same set of waveforms for low and high frequencies. An LHD with its noncollapsing property bypasses this problem. 
\MP{Note (perhaps in a footnotes) that it is possible to have separate surrogates for low and high frequency which are then joined smoothly as done for \texttt{SEOBNRv4\_ROM}.}

We build an initial training set with 128 waveforms sampled with an LHD in our 5-dimensional parameter space given by $q\in[1/3, 1]$, $\chi_1, \chi_2 \in [-0.5, 0.5]$ and $\Lambda_1, \Lambda_2 \in [0, 5000]$. In addition, we find empirically that the GPR uncertainty estimates (Eq.~\eqref{eq:var} and Eq.~\eqref{eq:errorrms} below) are largest at the corners of the parameter space, so we also sample the 32 corners. We construct our initial surrogate with these 160 waveforms shown in Fig.~\ref{fig:LHD}.

\begin{figure}[htb]
\centering
\includegraphics[width=0.49\textwidth]{trainingset3d.pdf}\\
\includegraphics[width=0.49\textwidth]{trainingset2d.pdf}
\caption{Projection of the sampled waveforms onto the three-dimensional subspace $\{q, \chi_1, \chi_2\}$ (top) and the two-dimensional subspace $\{\Lambda_1, \Lambda_2\}$ (bottom). The 160 blue circles were used for the initial training set (32 corners + 128 LHD points). The 200 red squares were generated using uncertainty sampling from the GPR error estimate (Eq.~\eqref{eq:errorrms}). The final 1000 green triangles were drawn uniformly from the parameter space.
}
\label{fig:LHD}
\end{figure}


With this initial surrogate model, we can choose new training-set samples by iteratively searching the parameter space for new points ${\bm x}$ that maximize some error criterion and update the surrogate with each new waveform. This method is sometimes referred to as uncertainty sampling~\cite{BrochuCoraDeFreitas2010}. The quantity that we use is the root-mean-squared error in the phase at the frequency nodes $Mf_j$ below $Mf=0.03$: \MP{Add reason why higher frequencies are not included in this error measure.}
\begin{equation}
\epsilon_{\rm RMS}(\bx) = \sqrt{\frac{1}{N} \sum_{ \substack{j \\ MF_j \le 0.03} } [\sigma_{\Delta\Phi_j}(\bx)]^2},
\label{eq:errorrms}
\end{equation}
where $\sigma_{\Delta\Phi_j}(\bx)$ is the GPR estimate (Eq.~\eqref{eq:var}) of the uncertainty in $\Delta\Phi(MF_j, \bx)$. There are many possible ways to construct a scalar that estimates overall waveform error. An alternative would be an approximation to the mismatch discussed in Sec.~\ref{sec:accuracy}. However, the quantity $\epsilon_{\rm RMS}(\bx)$ describes the most important quantity, phase, and is fast to compute. For an initial training set with $N$ samples, $\epsilon_{\rm RMS}(\bx)$ has $\sim N$ local maxima located in the voids between samples $\bx$. To find the global maximum, we use a basin hopping algorithm~\cite{WalesDoye1998, scipy:basinhopping} to avoid getting stuck in local maxima.

For low starting frequencies, EOB waveforms are expensive enough that we would like a method to efficiently choose $N_{\rm new}$ waveforms and evaluate them in parallel. We note that if we hold the hyperparameters $\btheta$ fixed, the GPR error estimate (Eq.~\eqref{eq:var}), and therefore $\epsilon_{\rm RMS}(\bx)$, only depends on the samples $\bx$ and not on the waveform data. The algorithm for choosing the $N_{\rm new}$ new points is as follows.


\begin{algorithm}[H]
\caption{Uncertainty Sampling}
\label{alg:uc}
\begin{algorithmic}[1]
\State {\bf Input:} Initial surrogate 
\For{$i = 1 \to N_{\rm New}$}

\State Construct the GPR error estimator (Eq.~\eqref{eq:errorrms}). In practice this can be done by specifying the samples $\bx$, the hyperparameters $\btheta$, and dummy data for $\by$ since Eq.~\eqref{eq:errorrms} does not depend on $\by$.

\State Find $({\bm x}_{\rm max}, \epsilon_{\rm rms})$ that maximizes Eq.~\eqref{eq:errorrms} over the parameter space.

\State Add ${\bm x}_{\rm max}$ to the list of samples: ${\bm x} = [{\bm x}, {\bm x}_{\rm max}]$.

\EndFor
\State {\bf Output:} Updated surrogate
\end{algorithmic}
\end{algorithm}



%{\scriptsize
%\begin{algorithm}[H]
%\caption{Empirical Interpolation (EI) Method}
%\label{alg:eim}
%\begin{algorithmic}[1]
%%\State {\bf Input:} $\{ \vec{e}_i \}_{i=1}^n$, $\{t_i\}_{i=1}^L$
%\State {\bf Input:} $\{ e_i \}_{i=1}^n$, $t := \{t_i\}_{i=1}^L$
%\vskip 10pt
%\State $i = \text{argmax} | e_1(t) |$ (\text{argmax} returns the largest entry of its argument). 
%\State Set $T_1 = t_i$
%\For{$j = 2 \to n$} 
%\State Build ${\cal I}_{j-1} [e_j](t)$ from (\ref{eq:empinterpX1})--(\ref{eq:empinterpX2})
%\State $\vec{r} = {\cal I}_{j-1} [e_j](t) -e_j(t)$
%\State $i = \text{argmax} |\vec{r}|$
%\State $T_j = t_i$
%\EndFor
%\vskip 10pt
%\State {\bf Output:} EI nodes $\{ T_i \}_{i=1}^n$,  interpolant operator ${\cal I}_n$
%\end{algorithmic}
%\end{algorithm}
%}


For $i = 1, \dots, N_{\rm new}$:
\begin{enumerate}

\item Construct the GPR error estimator $\epsilon_{\rm rms}(\bx)$ (Eq.~\eqref{eq:errorrms}). In practice this can be done by specifying the samples $\bx$, the hyperparameters $\btheta$, and dummy data for $\by$ since Eq.~\eqref{eq:errorrms} does not depend on $\by$.

\item Find $\bx_{\rm max}$ that maximizes $\epsilon_{\rm rms}(\bx)$ over the parameter space. 

\item Add $\bx_{\rm max}$ to the list of samples: ${\bm x} \to [{\bm x}, {\bm x}_{\rm max}]$.

\end{enumerate}
These $N_{\rm new}$ waveforms can now be evaluated in parallel. An updated surrogate can then be constructed with re-optimized hyperparameters $\btheta$ using the $N + N_{\rm new}$ waveforms.

Fig.~\ref{fig:LHD} shows the parameters of the 200 new waveforms chosen by the uncertainty sampling method. We note that the edges and faces of the parameter space are more often chosen than the inner region. This is likely a consequence of the Mat\'{e}rn GPR kernel used. Ref.~\cite{DoctorFarrHolz2017} found similar behavior when using a Mat\'{e}rn kernel instead of a squared exponential kernel. Fig.~\ref{fig:uncsamp} shows the RMS error $\epsilon_{\rm rms}$ maximized over the parameters $\bx$ for each new sample added to the training set. Increasing the number of waveforms from 160 to 360 decreases the estimated RMS phase error by a factor of $\sim 4$.

\begin{figure}[htb]
\centering
\includegraphics[width=0.49\textwidth]{uncertaintysampling.pdf}
\caption{Estimated 1--$\sigma$ RMS phase error $\epsilon_{\rm RMS}(\bx)$ (Eq.~\eqref{eq:errorrms}) maximized over parameter space $\bx$ as a function of the number of chosen parameters (blue curve). Also shown for comparison are the maximum RMS phase errors at the nodes where $MF_j \le 0.03$ for the surrogate compared to a test set of 1000 waveforms. Blue circle: surrogate constructed from the 160 waveforms in the initial training set. Red square: surrogate after adding the 200 waveforms chosen with uncertainty sampling. Green triangle: final surrogate after also adding 1000 more training set waveforms drawn uniformly from parameter space. }
\label{fig:uncsamp}
\end{figure}





%% ________________________________________________________
\section{Results}

\subsection{Accuracy}
\label{sec:accuracy}

The accuracy of the surrogate can be assessed by comparing it to a test set of waveforms with randomly sampled parameters. We evaluate an additional 1000 waveforms randomly sampled in parameter space using the same distribution discussed above to generate the last 1000 waveforms for the training set.

A common measure of the surrogate model accuracy is the mismatch between 
the surrogate model and the original EOB waveform.
The mismatch represents the loss in signal-to-noise ratio that would result 
from using the surrogate model instead of the original EOB waveform. 
It is defined by the deviation from a perfect overlap after aligning the two waveforms
using the time and phase free parameters $t_0$ and $\phi_0$:
\begin{equation}
\mathcal{M} = 1 - \max_{t_0, \phi_0} \frac{(h_{\rm EOB}, h_{\rm Sur})} {\sqrt{(h_{\rm EOB}, h_{\rm EOB}) (h_{\rm Sur}, h_{\rm Sur})}}.
\end{equation}
The inner product here  is the integral of the Fourier transformed waveforms $\tilde h(f)$ weighted by the noise power spectral 
density (PSD) $S_n(f)$ of the detector:
\begin{equation}
(h_1, h_2) = 4 \Re \int_{f_{\rm low}}^{f_{\rm high}} \frac{\tilde h_1(f) \tilde h^*_2(f)} {S_n(f)} df.
\end{equation}

In Fig.~\ref{fig:mismatch}, we show the distribution of mismatch $\mathcal{M}$ between our surrogate
and the $10^3$ randomly sampled EOB waveforms. We use the design sensitivity aLIGO PSD~\cite{Aasi:2013wya} and
a sampling rate of 4096~Hz. Our integration bounds are $f_{\rm low} = x$~Hz and the Nyquist frequency
$f_{\rm high} = 2048$~Hz. Because the surrogate can be rescaled with mass,
we show results for the smaller mass $M_B$ fixed at $1M_\odot$ or fixed at $2M_\odot$.
\red{[[This is all wrong. Fix it.]]}
We compare this mismatch to that of th

\begin{figure}[htb]
\centering
\includegraphics[width=0.49\textwidth]{mismatch.pdf}
\caption{Mismatches between the surrogate and 1000 test-set waveforms with randomly sampled parameters $\bx$. The total mass for each waveform is $M=2.8M_\odot$ and the PSD used was the aLIGO design sensitivity PSD. Histograms are shown for the initial surrogate with 160 waveforms (blue), the updated surrogate with 360 waveforms (red), and the final surrogate with 1360 waveforms (green). Also shown for comparison are the mismatches between TaylorF2 and the test set (black).}
\label{fig:mismatch}
\end{figure}

\begin{figure}[htb]
\centering
\includegraphics[width=0.49\textwidth]{htildemaxerror.pdf}
\caption{Amplitude and phase errors between the surrogate and the test-set waveform with the largest mismatch. Blue curve: initial surrogate with 160 waveforms. Red curve: updated surrogate with 360 waveforms. Green curve: final surrogate with 1360 waveforms. The vertical solid line is the gravitational-wave frequency at the Schwarzschild ISCO.}
\label{fig:maxmismatch}
\end{figure}

To demonstrate that we can correctly recover the behavior of the original unfiltered time-domain EOB waveform, we inverse Fourier transform the surrogate and compare it to the EOB waveform. In Fig.~\ref{fig:maxmismatchtd} we compare the surrogate to the test set waveform that had the largest mismatch. We align the two waveforms in time and phase by maximizing the overlap at early times in the interval $t/M \in [-2\times 10^5, -1\times 10^5]$ [[cite Read et al. for details.]]. The two waveforms are nearly identical except during the last $\sim 2$ cycles.

\begin{figure}[htb]
\centering
\includegraphics[width=0.49\textwidth]{hmaxerror.pdf}
\caption{Inverse Fourier transformed surrogate (green dashed) for the parameters $\bx$ that had the largest mismatch with the test-set EOB waveforms (black). The two waveforms are aligned in the time domain in the interval $t/M \in [-2\times 10^5, -1\times 10^5]$.}
\label{fig:maxmismatchtd}
\end{figure}


\subsection{Timing}

\MP{Add cost of complex exponential sincos function: look at profiling results}
\MP{Potentially, explore approximation to sincos for further speedup; need to guarantee that errors don't add up; range reduction is important for accuracy}

The surrogate evaluation time can be roughly decomposed into two times. The first is the time needed to evaluate the residuals $\Delta\ln(A)(F_j; \bx)$ and $\Delta\Phi(F_j; \bx)$ at each of the $N_A+N_\Phi$ interpolating nodes. Although optimizing the hyperparameters scales with the number of samples $n$ as $\mathcal{O}(n^3)$ due to the required matrix inversion, the evaluation time of a stored GPR scales as $\mathcal{O}(n)$. The evaluation time for the residuals will therefore be fixed cost of $\mathcal{O}(n(N_A+N_\Phi))$. The second part is the time needed to resample the final surrogate (Eq.[[ref]]) at uniformly spaced frequency samples in physical units beginning at a starting frequency $f_{\rm low}$.

In Fig.~\ref{fig:timing}, we show the surrogate evaluation time as a function of starting frequency $f_{\rm low}$. The surrogate has a fixed time of $\sim 50$~ms to evaluate the GPR at each node. The additional cost to resample the amplitude and phase at uniformly-spaced frequencies with spline interpolation dominates at starting frequencies below $\sim 30$~Hz. We also compare the evaluation time to several BNS waveform models that have been used for LIGO data analysis. We note that at $\sim 30$~Hz, the \texttt{SEOBNRv4T\_surrogate} model is comparable to the \texttt{SEOBNRv4\_ROM\_NRTidal} and \texttt{IMRPhenomD\_NRTidal} models that were used in the analysis of the GW170817 event~\cite{DietrichBernuzziTichy2017}. At lower frequencies, the surrogate is faster than all BNS waveform models except for TaylorF2. 

\begin{figure}[htb]
\centering
\includegraphics[width=0.49\textwidth]{timing.png}
\caption{Waveform evaluation time as a function of the waveform starting frequency. The time-domain waveforms
were sampled at 4096~Hz then Fourier transformed. The frequency-domain waveforms used the
same frequency samples as the Fourier transformed time-domain waveforms.}
\label{fig:timing}
\end{figure}

\subsection{Parameter estimation}

\MP{TODO: add 2D density plot of which quantities? masses, spins, lambdas?
or mass-ratio, eff spin, eff lambda?}


%% ________________________________________________________
\section{Discussion and future work}

We have constructed a fast frequency-domain surrogate of the most accurate BNS waveform currently available for data analysis. This aligned-spin model, \texttt{SEOBNRv4T}, incorporates the tidally induced $\ell=2$ and 3 multipole moments as well as the effect of dynamical tides as the excitation approaches the $\ell=2$ and 3 $f$-mode frequencies. We have achieved mismatches of no more than $7 \times 10^{-4}$ and phase errors of $\lesssim 1$~rad up to the merger frequency. These are sufficient to not bias results in any of the parameters.

The evaluation time has a flat cost of $\sim 0.02$~s to perform the GPR interpolation at each node. The rest of the time is spent resampling the waveform with spline interpolation. For a starting frequency of 10~Hz, this takes 0.3~s when the waveform is matched with data uniformly sampled at 4096Hz. For a starting frequency of 30~Hz, the total evaluation time is 0.04~s. For current parameter estimation codes, this is sufficient. However, one could further improve run times using reduced order quadrature~\cite{Antil2013, CanizaresFieldGair2013, CanizaresFieldGair2015, Smith:2016qas} which requires frequency-domain waveforms, or multi-band waveform interpolation~\cite{VinciguerraVeitchMandel2017}. Finally, we note that the production MCMC sampler used for the GW170817 analysis has an autocorrelation length of $\mathcal{O}(10^4)$ for aligned-spin BNS models, so significant improvements to the parameter estimation runtime can be made through better samplers. 

The hierarchical method presented here, where we begin with the analytic TaylorF2 reference model then make a surrogate of the residual, can be used to further improve the waveform model. For example, one could build a surrogate for numerical BNS simulations using \texttt{SEOBNRv4T\_surrogate} as the base model, and constructing a surrogate of the residual. Such a model would have the accuracy of EOB below $\sim 400$~Hz and the accuracy of numerical simulations for the last several cycles before merger. Current state of the art NR simulations have phase errors of several tenths of a radian over the last $\sim 20$ gravitational-wave cycles~\cite{DietrichHinderer2017, KiuchiKawaguchiKyutoku2017}. Using GPR and uncertainty sampling discussed above, one could optimally choose the waveform parameters for the numerical simulations, and run them in parallel to build the training set. Because the difference between \texttt{SEOBNRv4T} and numerical BNS simulations is exceptionally small, one would not need a high fractional accuracy for a surrogate of the difference, and 10--100 waveforms may be sufficient.

This model notably does not include two effects that can be important for systems with large spin. The first is the spin-induced quadrupole moment $Q$ that effects the waveform phase at 2PN order~\cite{Poisson1998}. The easiest way to correct for this effect is to simply add this 2PN term to the TaylorF2 reference waveform. One can use the universal relation between the quadrupole moment $Q$ and the tidal parameter $\Lambda$ as was done with the other NS matter parameters. Once this effect is incorporated into the EOB model, one could easily rebuild the surrogate presented in this paper to incorporate this effect. The quadrupole moment can be approximated in terms of the tidal parameter $\Lambda$ using a so-called universal relation~\cite{YagiYunes2013}.

The second effect is precession for non-aligned spins. Although none of the EOB models currently available include both tidal effects and precession, there are ways to boost this waveform model to a precessing frame and approximately incorporate precession. For example, Chatziioanno et al. have analytically solved the 2PN-accurate precession equations for generic spins~\cite{ChatziioannouKleinCornish2017a, ChatziioannouKleinCornish2017b}, with the exception of transitional precession which is unlikely for low mass-ratio and spin BNS systems. Using stationary uniform asymptotics, they can also analytically Fourier-transform the solution. Importantly, one is free to specify the frequency-domain amplitude and phase evolution, such as the \texttt{SEOBNRv4T\_surrogate} here, for the waveform in the co-precessing frame. This approach would provide a fast, accurate model for BNS systems with tides and generic spins.


%% ________________________________________________________
\begin{acknowledgments}

BL thanks the participants of the Bayesian Methods in Nuclear Physics workshop at the Institute for Nuclear Theory where many of the methods used here were discussed. We also thank Sylvain Marsat for correcting numerical instabilities in the original \texttt{SEOBNRv4T} code in \texttt{LAL}. The authors were supported by... 

\end{acknowledgments}


%% ________________________________________________________
\bibliography{paper,refs}  



\end{document}





%\begin{widetext}
%\begin{align}
%\Lambda_{2,\textrm{dyn}}^{\textrm{(A,B)}}(u) = &\frac{\lambda_{2}^{\textrm{(1,2)}}}{M^5} \left\{\frac{1}{4} + \frac{3}{4} \left(\frac{\omega^{(1,2)}_{f,2}}{2\Omega}\right)^2\left[\frac{4\Omega^2}{(\omega^{(1,2)}_{f,2})^2 - 4\Omega^2} + \frac{10}{3Q}\right.\right. \nonumber\\
%&\left.\left.+\sqrt{\frac{\pi}{3}}\frac{1}{\epsilon_{1,2}}\left[(1 + 2\mathcal{S}{(y_{1,2})})\cos{\left(\frac{\pi}{2}y^2_{1,2}\right)} - (1 + 2\mathcal{C}{(y_{1,2})})\sin{\left(\frac{\pi}{2}y_{1,2}^2\right)}\right]\right]\right\}\,,\\
%\Lambda_{3,\textrm{eff}}^{\textrm{(1,2)}} (u)= &\frac{\lambda_{3}^{\textrm{(1,2)}}}{M^7} \left\{\frac{3}{8} + \frac{(\omega^{(1,2)}_{f,3})^2}{\Omega^2}\left(\frac{25}{48 O_{1,2}} + \frac{5}{72}\frac{9\Omega^2}{(\omega^{(1,2)}_{f,3})^2-9\Omega^2}\right)\right. \nonumber\\
%&\left.+ O_{1,2}\left[\cos{(w^2)}\left(\frac{1}{2} +\mathcal{S}(sqrt2pi*w)\right) - \sin{(w^2)}\left(\frac{1}{2} + \mathcal{C}(sqrt2pi*w)\right)\right]\right\}\,,
%\end{align}
%\end{widetext}
%where $\lambda_{2}^{\textrm{(1,2)}}$ are the (constant) adiabatic values of the quadrupolar tidal polarizabilities, $\lambda_{3}^{\textrm{(1,2)}}$ are the (constant) adiabatic values of the octupolar tidal polarizabilities, $\omega^{(1,2)}_{f,2}$ are the quadrupolar $f$-mode frequencies, $\omega^{(1,2)}_{f,3}$ are the octupolar $f$-mode frequencies, $\mathcal{S}$ is the Fresnel sine integral, $\mathcal{C}$ is the Fresnel cosine integral, $\epsilon_{1,2} = 2^{19/3}\nu (M\omega^{(1,2)}_{f,2})^{5/3}/5$,   $y_{1,2}=\sqrt{3/(\pi \epsilon)} Q_{1,2}/5$, and $Q_{1,2} = 4- 2^{1/3}(M\omega^{(1,2)}_{f,2})^{5/3}u^{-5/2}$. 
%, $w_{1,2}=O_{1,2}/(4\,3^{2/3}\sqrt{10}(M\omega^{(1,2)}_{f,3})^{5/6}\sqrt{\nu})$, $O_{1,2}=5\sqrt{5\pi}/u^3(M\omega^{(1,2)}_{f,3})^{7/6}/(192\,3^{2/3}\sqrt{\nu})$.
%
